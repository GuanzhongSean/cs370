\documentclass{article}
\usepackage[a4paper,margin=1in]{geometry}
\usepackage{amsmath,amsfonts,amssymb,graphicx}
\title{\textbf{CS 370 Winter 2025: Assignment 2}}
\author{Jiaze Xiao \\ 20933691}
\date{\today}
\setlength{\parindent}{0pt}
\begin{document}

\maketitle

\section{Convergence of ODE Methods}
\subsection*{(a)}
Given
$$
    y'(t)=\dfrac{y(t)}{20}(10-y(t)),\quad y(0)=1/5
$$
To verify
$$
    y(t)=\dfrac{10}{49e^{-t/2}+1}
$$
We differentiate $ y(t) $ using the quotient rule:

Let $ u = 10 $ and $ v = 49e^{-t/2} + 1 $, so that:

$$
    y(t) = \frac{u}{v}
$$

Using the quotient rule:

$$
    y'(t) = \frac{u' v - u v'}{v^2}
$$

Since $ u = 10 $ is a constant, $ u' = 0 $, so:

$$
    y'(t) = \frac{-u v'}{v^2}
$$

Now, compute $ v' $:

$$
    v = 49e^{-t/2} + 1
$$

$$
    v' = -\frac{49}{2} e^{-t/2}
$$

Thus,

\begin{align*}
    y'(t) = & \frac{-10 \left(-\frac{49}{2} e^{-t/2} \right)}{(49e^{-t/2} + 1)^2} \\
    =       & \frac{10 \times \frac{49}{2} e^{-t/2}}{(49e^{-t/2} + 1)^2}          \\
    =       & \frac{245 e^{-t/2}}{(49e^{-t/2} + 1)^2}
\end{align*}

Now, compute the right-hand side of the ODE:

$$
    \frac{y}{20} (10 - y)
$$

First, substitute $ y(t) $:

\begin{align*}
    10 - y = & 10 - \frac{10}{49e^{-t/2} + 1}                             \\
    =        & 10 \left(1 - \frac{1}{49e^{-t/2} + 1} \right)              \\
    =        & 10 \left(\frac{49e^{-t/2} + 1 - 1}{49e^{-t/2} + 1} \right) \\
    =        & 10 \left(\frac{49e^{-t/2}}{49e^{-t/2} + 1} \right)
\end{align*}

Now compute:
\begin{align*}
    \frac{y}{20} (10 - y) = & \frac{1}{20} \times \frac{10}{49e^{-t/2} + 1} \times 10 \left(\frac{49e^{-t/2}}{49e^{-t/2} + 1} \right) \\
    =                       & \frac{5 \times 49e^{-t/2}}{(49e^{-t/2} + 1)^2}                                                          \\
    y'(t)=                  & \frac{245 e^{-t/2}}{(49e^{-t/2} + 1)^2}
\end{align*}

Since both sides are equal, the given function satisfies the ODE.

Finally, we need to verify $ y(0) = \frac{1}{5} $:

$$
    y(0) = \frac{10}{49e^{0} + 1} = \frac{10}{49 + 1} = \frac{10}{50} = \frac{1}{5}
$$

Thus, the initial condition holds.

\newpage
\section{Higher Order and Systems of ODEs}
Define:
$$ v_x = \frac{dx}{dt} $$
$$ v_y = \frac{dy}{dt} $$

Now, rewrite the second-order equations:

$$
    \frac{d v_x}{dt} = \dfrac{d^2x}{dt^2} = - \frac{x}{(x^2 + y^2)^{3/2}}
$$

$$
    \frac{d v_y}{dt} = \dfrac{d^2y}{dt^2} = - \frac{y}{(x^2 + y^2)^{3/2}}
$$

Define the state vector:

$$
    \mathbf{u} = \begin{bmatrix} x \\ y \\ v_x \\ v_y \end{bmatrix}
$$

Then, the system can be written in the form:

$$
    \frac{d \mathbf{u}}{dt} = F(t, \mathbf{u})
$$

where:

$$
    F(t, \mathbf{u}) =
    \begin{bmatrix}
        v_x                           \\
        v_y                           \\
        - \frac{x}{(x^2 + y^2)^{3/2}} \\
        - \frac{y}{(x^2 + y^2)^{3/2}}
    \end{bmatrix}
$$

Given initial values:

$$
    x(0) = 0.4, \quad y(0) = 0, \quad v_x(0) = 0, \quad v_y(0) = 2
$$

and pass them into vector $\mathbf{u}$:

$$
    \mathbf{u}(0) = \begin{bmatrix} 0.4 \\ 0 \\ 0 \\ 2 \end{bmatrix}
$$

As a result, the first-order system of ODEs is:

$$
    \frac{d}{dt} \begin{bmatrix} x \\ y \\ v_x \\ v_y \end{bmatrix} =
    \begin{bmatrix}
        v_x                           \\
        v_y                           \\
        - \frac{x}{(x^2 + y^2)^{3/2}} \\
        - \frac{y}{(x^2 + y^2)^{3/2}}
    \end{bmatrix}
$$

with initial conditions:

$$
    \mathbf{u}(0) = \begin{bmatrix} 0.4 \\ 0 \\ 0 \\ 2 \end{bmatrix}
$$

\newpage
\section{Forward Euler and Improved Euler Methods}
The Forward Euler method is:
$$
    y_{n+1} = y_n + h f(x_n, y_n)
$$
where \( f(x, y) = x + y \).

At \( x_1 = 0.5 \),
$$ y_0 = 2 $$
$$ f(0,2) = 0 + 2 = 2 $$
$$
    y_1 = 2 + 0.5 \times 2 = 2 + 1 = 3
$$

At \( x_2 = 1.0 \),
$$ f(0.5, 3) = 0.5 + 3 = 3.5 $$
$$
    y_2 = 3 + 0.5 \times 3.5 = 3 + 1.75 = 4.75
$$

The Improved Euler method is:

$$
    k_1 = hf(x_n, y_n)
$$
$$
    k_2 = hf(x_n + h, y_n + k_1)
$$
$$
    y_{n+1} = y_n + \frac{k_1}{2} + \frac{k_2}{2}
$$

At \( x_1 = 0.5 \),
$$ y_0 = 2 $$
$$ k_1 = 0.5\times f(0,2) = 0.5\times  2 = 1 $$
$$ k_2 = 0.5\times f(0.5, 2 + 1) = 0.5\times 3.5 = 1.75 $$
$$ y_1 = 2 + \frac{1}{2}+ \frac{1.75}{2}= 3.375 $$

At \( x_2 = 1.0 \),
$$ k_1 = 0.5\times f(0.5, 3.375) = 0.5 \times 3.875 = 1.9375 $$
$$ k_2 = 0.5 \times f(1.0, 3.375 + 1.9375) = 0.5 \times 6.3125 = 3.15625 $$
$$ y_2 = 3.375 + \frac{1.9375}{2} + \frac{3.15625}{2}=5.921875 $$

The exact solution is:

$$
    y(x) = 3e^x - x - 1
$$

For \( x = 0.5 \):

$$
    y(0.5) = 3e^{0.5} - 0.5 - 1 \approx 3.446
$$

For \( x = 1.0 \):

$$
    y(1.0) = 3e^{1.0} - 1.0 - 1 \approx 6.155
$$

$$
    \begin{array}{|c|c|c|c|c|c|}
        \hline
        x   & \text{Exact Solution} & \text{Forward Euler} & \text{FE Error} & \text{Improved Euler} & \text{IE Error} \\
        \hline
        0.5 & 3.446                 & 3.000                & 0.446           & 3.375                 & 0.071           \\
        1.0 & 6.155                 & 4.750                & 1.405           & 5.922                 & 0.233           \\
        \hline
    \end{array}
$$

\newpage
\section{Local Truncation Error}

\newpage
\section{Stability}

\end{document}
