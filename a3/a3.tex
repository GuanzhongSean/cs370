\documentclass{article}
\usepackage[a4paper,margin=1in]{geometry}
\usepackage{amsmath,amsfonts,amssymb,graphicx}
\title{\textbf{CS 370 Winter 2025: Assignment 3}}
\author{Jiaze Xiao \\ 20933691}
\date{\today}
\setlength{\parindent}{0pt}
\begin{document}

\maketitle

\section{DFT of a sine signal}
We have
$$
    f_n =\sin\bigl(\tfrac{8\pi n}{N}\bigr),
    \quad n=0,1,\dots,N-1,
$$

Since
$$
    \sin\theta =\frac{1}{2i}
    \bigl(e^{i\theta} - e^{-i\theta}\bigr).
$$
Hence
$$
    F_k
    =
    \frac{1}{N}\sum_{n=0}^{N-1} f_n W^{-nk}
    =
    \frac{1}{N}\sum_{n=0}^{N-1}
    \frac{1}{2i}\Bigl(
    e^{i\tfrac{8\pi n}{N}}
    -
    e^{-i\tfrac{8\pi n}{N}}
    \Bigr)
    W^{-nk}.
$$
Factor out $N/(2i\pi)$ and combine exponents by noting that
$
    e^{i(8\pi n/N)} = W^{4n}
$

$$
    F_k
    =
    \frac{1}{2iN}\sum_{n=0}^{N-1}
    \Bigl(
    W^{4n}W^{-nk}
    -
    W^{-4n}W^{-nk}
    \Bigr)
    =
    \frac{1}{2iN}\sum_{n=0}^{N-1}
    \Bigl(
    W^{n(4-k)}
    -
    W^{n(N -4 - k)}
    \Bigr).
$$

We use the fact that
$$
    \sum_{n=0}^{N-1} W^{n(k-l)}
    =
    \begin{cases}
        N, & \text{if}\ k=l,   \\
        0, & \text{otherwise}.
    \end{cases}
$$
Therefore, each of the two sums in parentheses is either $N$ or $0,$ depending on whether the exponent $(4-k)$ (respectively $N -4 - k$) is $0.$

\begin{itemize}
    \item Case 1: $k=4$. In this case, the first sum is $N$, and the second sum is $0$.
    \item Case 2: $k=N-4$. Then the second sum is $N$, while the first sum is $0$.
    \item All other $k$ give zero for both sums.
\end{itemize}

As a result,
$$
    \boxed{
        F_k
        =
        \begin{cases}
            \displaystyle
            -\tfrac{i}{2},
            \quad & k = 4,           \\
            \displaystyle
            +\tfrac{i}{2},
            \quad & k = N-4,         \\
            0,    & \text{otherwise}
        \end{cases}
    }.
$$

\newpage
\section{DFT of special signals}
\subsection*{(a)}
Plug $f_n = W^{5n}$ into the DFT sum:
$$
    F_k
    =\frac{1}{N}\sum_{n=0}^{N-1} W^{5n}W^{-nk}
    =\frac{1}{N}\sum_{n=0}^{N-1} W^{n(5-k)}.
$$
The sum is $N$ only if $k=5$. Thus,
$$
    \boxed{
        F_k
        =
        \begin{cases}
            1, & k=5,    \\
            0, & k\neq 5
        \end{cases}
    }.
$$

\subsection*{(b)}
$$
    F_k
    =\frac1N\sum_{n=0}^{N-1}
    (-1)^nW^{-nk}
    =\frac1N\sum_{n=0}^{N-1}
    \Bigl[-W^{-k}\Bigr]^{n}
    =\frac1N\sum_{n=0}^{N-1}R^n,
    \quad
    R = -W^{-k}.
$$
Since $N$ is odd, $R^N=(-1)^N(W^{-k})^N=(-1)^N=-1.$. Thus,
$$
    \sum_{n=0}^{N-1}R^n
    =\frac{R^N-1}{R-1}
    =\frac{(-1)-1}{R-1}
    =\frac{2}{1-R}
    =\frac{2}{1-(-W^{-k})}
    =\frac{2}{1+W^{-k}}.
$$
Therefore,
$$
    \boxed{
    F_k
    = \frac{1}{N}\frac{2}{1+W^{-k}}
    =\frac{2}{N\bigl(1 + e^{-2\pi i k/N}\bigr)}
    }.
$$
This is valid for every $k$.  In particular, there is no single “spike” in frequency; instead all bins $k$ are generally nonzero when $N$ is odd.

\subsection*{(c)}
Observe that $(-1)^n = e^{i\pi n} = W^{\frac{nN}{2}}.$  Then
$$
    F_k
    =\frac1N\sum_{n=0}^{N-1}
    (-1)^nW^{-nk}
    =\frac1N\sum_{n=0}^{N-1} W^{\frac{nN}{2}}W^{-nk}
    =\frac1N\sum_{n=0}^{N-1} W^{n\left(\frac{N}{2}-k\right)}.
$$
The sum is $N$ only if $k=\frac{N}{2}$. Thus,
$$
    \boxed{
        F_k
        =
        \begin{cases}
            1, & k=\frac{N}{2}, \\
            0, & k\neq 5
        \end{cases}
    }.
$$

\newpage
\section{DFT of a real and even vector}
First, compute \(F_{N-k}\):
\[
    F_{N-k}
    = \frac{1}{N}\sum_{n=0}^{N-1} f_nW^{-n(N-k)}
    = \frac{1}{N}\sum_{n=0}^{N-1} f_nW^{nk}.
\]
Now make the change of index \(m = N - n\).  Because \(f\) is even, \(f_{N-m} = f_m\).  One finds
\[
    F_{N-k}
    = \frac{1}{N}\sum_{m=0}^{N-1} f_{N-m} W^{(N-m)k}
    = \frac{1}{N}\sum_{m=0}^{N-1} f_m W^{-m k}=F_k.
\]
Hence \(F_{N-k} = F_k\), so \(F\) is an even (symmetric) vector. And since $f$ is real, its DFT is conjugate symmetric, we have:
$$
    F_k=\overline{F_{N-k}}
$$
And $F_k=F_{N-k}$, we can conclude that each $F_k$ is real. Combining both parts proves that when \(f\) is a real and even vector, its DFT \(F\) is also real and even.

\newpage
\section{FFT}
Stage 1 (\(N=8, W=e^{\frac{\pi i}{4}}\)),

\(n=0\):
\[
    g_0 = \tfrac12(f_0 + f_4)
    = \tfrac12(-1 + 5) = 2,
    \quad
    h_0 = \tfrac12(f_0 - f_4) W^{0}
    = \tfrac12(-1 - 5)\cdot 1
    = -3.
\]
\(n=1\):
\[
    g_1 = \tfrac12(f_1 + f_5)
    = \tfrac12(2 + 2) = 2,
    \quad
    h_1 = \tfrac12(f_1 - f_5) W^{-1}
    = \tfrac12(2 - 2) W^{-1}
    = 0.
\]
\(n=2\):
\[
    g_2 = \tfrac12(f_2 + f_6)
    = \tfrac12(3 + 3) = 3,
    \quad
    h_2 = \tfrac12(f_2 - f_6) W^{-2}
    = \tfrac12(3 - 3) (\cdots)
    = 0.
\]
\(n=3\):
\[
    g_3 = \tfrac12(f_3 + f_7)
    = \tfrac12(4 + 4) = 4,
    \quad
    h_3 = \tfrac12(f_3 - f_7) W^{-3}
    = \tfrac12(4 - 4) (\cdots)
    = 0.
\]

Thus,
\[
    g = [ 2, 2, 3, 4 ],
    \quad
    h = [ -3, 0, 0, 0 ].
\]

Stage 2 (\(N=4, W=e^{\frac{\pi i}{2}}\)),
\begin{itemize}
    \item Let $f_{top}=g = [ 2, 2, 3, 4 ]$,\\
          $n=0$,
          $$
              g_0 = \tfrac12(f_0 + f_2)
              = \tfrac12(2+3) = 2.5,
              \quad
              h_0 = \tfrac12(f_0 - f_2) W^{0}
              = \tfrac12(2-3)\cdot 1
              = -0.5.
          $$
          $n=1$,
          $$
              g_1 = \tfrac12(f_1 + f_3)
              = \tfrac12(2+4) = 3,
              \quad
              h_1 = \tfrac12(f_1 - f_3) W^{-1}
              = \tfrac12(2-4)\cdot (-i)
              = i.
          $$
          Thus,
          $$
              g_{top}=[2.5,3],
              \quad
              h_{top}=[-0.5,i].
          $$
    \item Let $f_{bottom}=h = [ -3,0,0,0 ]$,\\
          $n=0$,
          $$
              g_0 = \tfrac12(f_0 + f_2)
              = \tfrac12(-3+0) = -1.5,
              \quad
              h_0 = \tfrac12(f_0 - f_2) W^{0}
              = \tfrac12(-3-0)\cdot 1
              = -1.5.
          $$
          $n=1$,
          $$
              g_1 = \tfrac12(f_1 + f_3)
              = \tfrac12(0+0) = 0,
              \quad
              h_1 = \tfrac12(f_1 - f_3) W^{-1}
              = \tfrac12(0-0)\cdot (-i)
              = 0.
          $$
          Thus,
          $$
              g_{bottom}=[-1.5,0],
              \quad
              h_{bottom}=[-1.5,0].
          $$
\end{itemize}

Stage 3 (\(N=2, W=e^{\pi i}\)),
\begin{itemize}
    \item Let $f_{1st}=g_{top}=[2.5,3]$,
          $$
              g_0 = \tfrac12(f_0 + f_1)
              = \tfrac12(2.5+3) = \frac{11}{4},
              \quad
              h_0 = \tfrac12(f_0 - f_1) W^{0}
              = \tfrac12(2.5-3)\cdot 1
              = -\frac{1}{4}.
          $$
    \item Let $f_{2nd}=h_{top}=[-0.5,i]$,
          $$
              g_0 = \tfrac12(f_0 + f_1)
              = \tfrac12(-0.5+i) = -\frac{1}{4}+\frac{i}{2},
              \quad
              h_0 = \tfrac12(f_0 - f_1) W^{0}
              = \tfrac12(-0.5-i)\cdot 1
              = -\frac{1}{4}-\frac{i}{2}.
          $$
    \item Let $f_{3rd}=g_{bottom}=[-1.5,0]$,
          $$
              g_0 = \tfrac12(f_0 + f_1)
              = -\frac{3}{4},
              \quad
              h_0 = \tfrac12(f_0 - f_1) W^{0}
              = -\frac{3}{4}.
          $$
    \item Let $f_{4th}=h_{bottom}=[-1.5,0]$,
          $$
              g_0 = \tfrac12(f_0 + f_1)
              = -\frac{3}{4},
              \quad
              h_0 = \tfrac12(f_0 - f_1) W^{0}
              = -\frac{3}{4}.
          $$
\end{itemize}
Unscramble the cofficients we get:
$$
    \boxed{
        F=[\frac{11}{4}, -\frac{3}{4}, -\frac{1}{4}+\frac{i}{2}, -\frac{3}{4}, -\frac{1}{4}, -\frac{3}{4}, -\frac{1}{4}-\frac{i}{2}, -\frac{3}{4}]
    }
$$

\end{document}
