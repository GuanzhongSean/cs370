\documentclass{article}
\usepackage[a4paper,margin=1in]{geometry}
\usepackage{amsmath,amsfonts,amssymb,graphicx}
\title{\textbf{CS 370 Winter 2025: Assignment 3}}
\author{Jiaze Xiao \\ 20933691}
\date{\today}
\setlength{\parindent}{0pt}
\begin{document}

\maketitle

\section{DFT of a sine signal}
We have
$$
    f_n =\sin\bigl(\tfrac{8\pi n}{N}\bigr),
    \quad n=0,1,\dots,N-1,
$$

Since
$$
    \sin\theta =\frac{1}{2i}
    \bigl(e^{i\theta} - e^{-i\theta}\bigr).
$$
Hence
$$
    F_k
    =
    \frac{1}{N}\sum_{n=0}^{N-1} f_n W^{-nk}
    =
    \frac{1}{N}\sum_{n=0}^{N-1}
    \frac{1}{2i}\Bigl(
    e^{i\tfrac{8\pi n}{N}}
    -
    e^{-i\tfrac{8\pi n}{N}}
    \Bigr)
    W^{-nk}.
$$
Factor out $N/(2i\pi)$ and combine exponents by noting that
$
    e^{i(8\pi n/N)} = W^{4n}
$

$$
    F_k
    =
    \frac{1}{2iN}\sum_{n=0}^{N-1}
    \Bigl(
    W^{4n}W^{-nk}
    -
    W^{-4n}W^{-nk}
    \Bigr)
    =
    \frac{1}{2iN}\sum_{n=0}^{N-1}
    \Bigl(
    W^{n(4-k)}
    -
    W^{n(N -4 - k)}
    \Bigr).
$$

We use the fact that
$$
    \sum_{n=0}^{N-1} W^{n(k-l)}
    =
    \begin{cases}
        N, & \text{if}\ k=l,   \\
        0, & \text{otherwise}.
    \end{cases}
$$
Therefore, each of the two sums in parentheses is either $N$ or $0,$ depending on whether the exponent $(4-k)$ (respectively $N -4 - k$) is $0.$

\begin{itemize}
    \item Case 1: $k=4$. In this case, the first sum is $N$, and the second sum is $0$.
    \item Case 2: $k=N-4$. Then the second sum is $N$, while the first sum is $0$.
    \item All other $k$ give zero for both sums.
\end{itemize}

As a result,
$$
    \boxed{
        F_k
        =
        \begin{cases}
            \displaystyle
            -\tfrac{i}{2},
            \quad & k = 4,           \\
            \displaystyle
            +\tfrac{i}{2},
            \quad & k = N-4,         \\
            0,    & \text{otherwise}
        \end{cases}
    }.
$$

\newpage
\section{DFT of special signals}
\subsection*{(a)}
Plug $f_n = W^{5n}$ into the DFT sum:
$$
    F_k
    =\frac{1}{N}\sum_{n=0}^{N-1} W^{5n}W^{-nk}
    =\frac{1}{N}\sum_{n=0}^{N-1} W^{n(5-k)}.
$$
The sum is $N$ only if $k=5$. Thus,
$$
    \boxed{
        F_k
        =
        \begin{cases}
            1, & k=5,    \\
            0, & k\neq 5
        \end{cases}
    }.
$$

\subsection*{(b)}
$$
    F_k
    =\frac1N\sum_{n=0}^{N-1}
    (-1)^nW^{-nk}
    =\frac1N\sum_{n=0}^{N-1}
    \Bigl[-W^{-k}\Bigr]^{n}
    =\frac1N\sum_{n=0}^{N-1}R^n,
    \quad
    R = -W^{-k}.
$$
Since $N$ is odd, $R^N=(-1)^N(W^{-k})^N=(-1)^N=-1.$. Thus,
$$
    \sum_{n=0}^{N-1}R^n
    =\frac{R^N-1}{R-1}
    =\frac{(-1)-1}{R-1}
    =\frac{2}{1-R}
    =\frac{2}{1-(-W^{-k})}
    =\frac{2}{1+W^{-k}}.
$$
Therefore,
$$
    \boxed{
    F_k
    = \frac{1}{N}\frac{2}{1+W^{-k}}
    =\frac{2}{N\bigl(1 + e^{-2\pi i k/N}\bigr)}
    }.
$$
This is valid for every $k$.  In particular, there is no single “spike” in frequency; instead all bins $k$ are generally nonzero when $N$ is odd.

\subsection*{(c)}
Observe that $(-1)^n = e^{i\pi n} = W^{\frac{nN}{2}}.$  Then
$$
    F_k
    =\frac1N\sum_{n=0}^{N-1}
    (-1)^nW^{-nk}
    =\frac1N\sum_{n=0}^{N-1} W^{\frac{nN}{2}}W^{-nk}
    =\frac1N\sum_{n=0}^{N-1} W^{n\left(\frac{N}{2}-k\right)}.
$$
The sum is $N$ only if $k=\frac{N}{2}$. Thus,
$$
    \boxed{
        F_k
        =
        \begin{cases}
            1, & k=\frac{N}{2}, \\
            0, & k\neq 5
        \end{cases}
    }.
$$

\newpage
\section{DFT of a real and even vector}

\newpage
\section{FFT}

\end{document}
